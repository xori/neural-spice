
\documentclass[journal]{IEEEtran}
\usepackage{lmodern}
\usepackage{amsfonts}
%\usepackage{hyperref}

\usepackage{cite}
\ifCLASSINFOpdf
  \usepackage[pdftex]{graphicx}
  \graphicspath{{./img/}}
  \DeclareGraphicsExtensions{.pdf,.jpeg,.png}
\else
\fi
\usepackage{array}
\usepackage{url}
\hyphenation{op-tical net-works semi-conduc-tor av-er-age at-tribute}


\begin{document}
\title{Anaylisis of Genetic Algorithms Optimizing Topological Layout and Synaptic Weights }

\author{Taras~Mychaskiw(xxxxxxx)~\textless{}tm07qx@brocku.ca\textgreater\\%
Evan~Verworn~(4582938)~\textless{}ev09qz@brocku.ca\textgreater% <-this % stops a space
}

\maketitle

\begin{abstract}
This report is to show the uses of a Self Organizing Feature Map in cases of clustering. The Self Organizing Maps are then analyzed using density based clustering
modelling and cluster validation using the Dunn Index.
\end{abstract}

\IEEEpeerreviewmaketitle

\section{Introduction}
\IEEEPARstart{T}{he} purpose of this assignment is for use to show that we understand
Self Organizing Maps and methods of performing analysis on the results. Self Organizing Maps also are useful for reducing the curse of dimensionality problem
that arises within other ANNs by eliminating data points that are too similar or 
outliers. 

\section{Background}
  \subsection{Neural Networks}
  \subsection{Genetic Algoritms}

\section{Methodology}
  \subsection{Mutations}
    Our custom mutations.
  \subsection{Crossovers}
    Our custom crossovers.

\section{Experiments}
  \subsection{Connect 4}
    Description of Connect 4 Problem
  \subsection{Quality of Wines}
    Description of Wines Problem
  \subsection{Experiment Setup}
    Parameters Used.
  
\section{Analysis}
Results

\section{Conclusion}
Things could have done better.

We believe this is due to crossovers used.

Blah-de-Blah.


% references section

% can use a bibliography generated by BibTeX as a .bbl file
% BibTeX documentation can be easily obtained at:
% http://www.ctan.org/tex-archive/biblio/bibtex/contrib/doc/
% The IEEEtran BibTeX style support page is at:
% http://www.michaelshell.org/tex/ieeetran/bibtex/
%\bibliographystyle{IEEEtran}
% argument is your BibTeX string definitions and bibliography database(s)
%\bibliography{IEEEabrv,../bib/paper}
%
% <OR> manually copy in the resultant .bbl file
% set second argument of \begin to the number of references
% (used to reserve space for the reference number labels box)
\begin{thebibliography}{1}

\bibitem{SOM_KO}
Kohonen, T. (2001). Self-Organizing Maps. Third, extended edition. Springer, Berlin.

\bibitem{Dunn}
Dunn, J. (1974). "Well separated clusters and optimal fuzzy partitions". Journal of Cybernetics 4

\end{thebibliography}

% You can push biographies down or up by placing
% a \vfill before or after them. The appropriate
% use of \vfill depends on what kind of text is
% on the last page and whether or not the columns
% are being equalized.

%\vfill

% Can be used to pull up biographies so that the bottom of the last one
% is flush with the other column.
%\enlargethispage{-5in}



% that's all folks
\end{document}

